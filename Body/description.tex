
\section{Descrição do Programa}

\subsection{Programa Tarifa Zero}
Essa proposta de avaliação experimental se baseia no Programa Tarifa Zero de São Caetano do Sul, cidade de 165 mil habitantes da região metropolitana de São Paulo. Segundo o prefeito da cidade, José Auricchio Júnior, que instituiu a política por decreto em novembro de 2023, o objetivo é melhorar a mobilidade urbana na cidade, aumentar a empregabilidade da população e fomentar a atividade econômica PREF_SCS. 

Subsídios de transporte público são um tema comum na literatura de Economia Urbana, e outros trabalham mostram como a política pode atender aos objetivos propostos. Há indícios de que subsídios como a Tarifa Zero aumentam o raio de busca de indivíduos que procuram emprego, o que aumenta a taxa de empregabilidade e pode incentivar trabalhadores a buscarem firmas mais distantes com melhor remuneração (BETTER_FIRMS, NO_DAM). No que diz respeito à mobilidade urbana, o programa aumenta o acesso ao transporte público, o que permite que mais pessoas se locomovam de ônibus pela cidade e incentiva o uso desse modal frente a outras alternativas, como carros, que tendem a gerar mais lentidão e congestionamentos. 

Os artigos não observam conclusivamente um canal pelo qual a política aquece a atividade econômica para além da empregabilidade, mas análises da prefeitura de São Caetano, embora sem rigor econométrico, concluem que o número de estabelecimentos comerciais aumentou em áreas servidas pelas linhas gratuitas PREF_2024, já que o programa aumenta a circulação de pessoas no transporte e, consequentemente, o consumo nessas regiões. Isso é algo que nosso experimento pode investigar. Há uma evidência adicional, para além dos objetivos da política de São Caetano, encontrada em alguns artigos: a tarifa zero também reduz emissões de gás carbônico (NO_DAM).

O programa tem como público-alvo todos os usuários do transporte municipal de São Caetano do Sul. A política pública funciona da seguinte forma: qualquer pessoa, sem que seja necessário comprovar residência na cidade ou o cumprimento de critérios socioeconômicos, pode viajar gratuitamente em todos os ônibus da rede municipal, de forma ilimitada. Ônibus e trens metropolitanos, operados pelo Governo do Estado, não estão inclusos no programa, mesmo que passem por São Caetano.

A concessionária que opera o transporte é remunerada pela prefeitura com base na quilometragem percorrida pelos ônibus. A política transportou, entre novembro de 2023 e dezembro de 2024, 22 milhões de passageiros, segundo a prefeitura PREF_2024. A frota que atende o programa é composta por 63 ônibus, e transporta uma média diária de 72 mil passageiros PREF_2023. O custo estimado da política é de 2,9 milhões de reais mensais LEI_PROG.

\subsection{Proposta de Expansão do Programa}

Tendo isso em vista, o grupo propõe uma política pública de Tarifa Zero em transporte municipal em moldes similares à política implementada em São Caetano. A política pública proposta é de Tarifa Zero em transporte metropolitano municipal (ônibus), com o objetivo inicial de melhorar a mobilidade urbana, porém com objetivos secundários de trazer benefícios ao mercado de trabalho e de reduzir a emissão de gás carbônico BETTER_FIRMS, NO_DAM, PREF_2024.

A política pública seria aplicada nos municípios no Estado de São Paulo (645 municípios) fora da Região Metropolitana de São Paulo (RMSP), que conta com transporte público intermunicipal (39 municípios). O transporte inter-urbano fora da RMSP é privado, fazendo com que subsídios escapem do escopo da política pública proposta. Além disso, os municípios que já adotaram a Tarifa Zero estão principalmente na RMSP, e serão desconsiderados da análise. Assim, o público alvo seriam os 606 municípios restantes no Estado de São Paulo. A  Tarifa Zero seria implementada em ondas dentre os municípios, de forma a separar a amostra em grupo tratado e em um grupo que será de controle enquanto não for tratado. A escolha dos municípios tratados será por randomização.

A política funcionaria nos mesmos moldes que a implementação do Programa Tarifa Zero de São Caetano do Sul: qualquer pessoa pode utilizar transporte municipal de forma gratuita, ilimitadamente. Sem restrição pecuniária para o uso do transporte público, há mais incentivo para seu uso e uma consequente redução às barreiras da mobilidade urbana, o problema elencado.
