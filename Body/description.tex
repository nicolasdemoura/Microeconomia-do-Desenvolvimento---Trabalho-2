
\section{Descrição do Programa}

\subsection{Programa Tarifa Zero}
A proposta de avaliação experimental tem como foco o Programa Tarifa Zero implementado no município de São Caetano do Sul, na Região Metropolitana de São Paulo. Instituída por decreto em novembro de 2023, a política garante gratuidade integral no transporte coletivo municipal. O programa atende uma população de aproximadamente 165 mil habitantes e, segundo a prefeitura local, tem como metas centrais melhorar a mobilidade urbana, ampliar as oportunidades de emprego e impulsionar a atividade econômica local \cite{PREF_SCS}.

Do ponto de vista econômico, o programa busca enfrentar falhas de mercado relacionadas à provisão subótima de bens públicos e à presença de externalidades positivas não internalizadas no uso do transporte coletivo. Ao reduzir o custo marginal de uso a zero, a política corrige a subutilização típica de bens públicos como o transporte coletivo, cujo uso socialmente desejável é maior do que o que se observa em mercados com tarifa. Além disso, ao incentivar a substituição do transporte individual motorizado (com maior impacto ambiental e congestionamento) pelo transporte público, a política gera externalidades positivas em termos de redução de poluição, melhora da fluidez viária e acesso ampliado a serviços e empregos, especialmente para as camadas mais vulneráveis da população.

O funcionamento da política consiste em subsidiar integralmente as tarifas do transporte coletivo municipal, de modo que os usuários passem a utilizar o serviço sem custo direto. Ao eliminar a barreira financeira ao uso do transporte público, o programa expande o raio de busca por oportunidades de emprego, particularmente entre trabalhadores de baixa renda, o que pode reduzir o desemprego friccional e melhorar o pareamento entre trabalhadores e firmas \cite{BETTER_FIRMS, NO_DAM}. A política também altera o equilíbrio modal, tornando o transporte coletivo mais competitivo em relação ao transporte privado, o que tende a reduzir o número de veículos em circulação, desafogando o trânsito e diminuindo tempos de deslocamento.

Ainda que o vínculo entre tarifa zero e dinamismo econômico seja menos consolidado na literatura, análises iniciais da prefeitura sugerem um aumento no número de estabelecimentos comerciais em áreas atendidas pelas linhas gratuitas, possivelmente em razão da maior circulação de pessoas e do estímulo à demanda local \cite{PREF_2024} — um aspecto adicional que poderá ser explorado na avaliação proposta. Há também evidências de que a tarifa zero pode contribuir para a redução das emissões de gases poluentes, reforçando os benefícios ambientais da política \cite{NO_DAM}.

O programa tem como público-alvo todos os usuários do transporte municipal de São Caetano do Sul. A política pública funciona da seguinte forma: qualquer pessoa, sem que seja necessário comprovar residência na cidade ou o cumprimento de critérios socioeconômicos, pode viajar gratuitamente em todos os ônibus da rede municipal, de forma ilimitada. Ônibus e trens metropolitanos, operados pelo Governo do Estado, não estão inclusos no programa, mesmo que passem por São Caetano.

A concessionária que opera o transporte é remunerada pela prefeitura com base na quilometragem percorrida pelos ônibus. A política transportou, entre novembro de 2023 e dezembro de 2024, 22 milhões de passageiros, segundo a prefeitura \cite{PREF_2024}. A frota que atende o programa é composta por 63 ônibus, e transporta uma média diária de 72 mil passageiros \cite{PREF_2023}. O custo estimado da política é de 2{,}9 milhões de reais mensais \cite{LEI_PROG}.

\subsection{Proposta de Expansão}

Tendo como referência a experiência de São Caetano do Sul, propõe-se a implementação de uma política pública estadual de incentivo à adoção da \textit{Tarifa Zero} no transporte coletivo municipal. A proposta tem como objetivo central a ampliação do acesso à mobilidade urbana, com efeitos esperados sobre o dinamismo do mercado de trabalho e a redução das emissões de gases poluentes \cite{BETTER_FIRMS, NO_DAM, PREF_2024}.

A política será direcionada aos 606 municípios do Estado de São Paulo que se encontram fora da Região Metropolitana de São Paulo (RMSP). Nessa região, o transporte público tem forte caráter intermunicipal e é regulado pela Empresa Metropolitana de Transportes Urbanos (EMTU). Já nas demais regiões do estado, o transporte intermunicipal é majoritariamente privado e, portanto, fora do escopo direto da política proposta. Municípios que já adotaram a gratuidade no transporte municipal também serão excluídos da amostra inicial. A implementação seguirá um modelo de adoção gradual, em ondas sucessivas, possibilitando a formação de grupos de comparação entre localidades tratadas e não tratadas, a partir de processos de randomização, a ser detalhado nas próximas seções.

Nos moldes do programa em vigor em São Caetano, a proposta prevê que qualquer pessoa poderá utilizar gratuitamente os ônibus do sistema municipal, sem necessidade de comprovação de residência ou renda, e sem limite de viagens. Ao eliminar a barreira do custo, espera-se ampliar significativamente o uso do transporte coletivo e, com isso, mitigar desigualdades de acesso a serviços e oportunidades.

De acordo com o Art. 18 da Lei nº 12.587/2012, cabe aos municípios planejar, executar e avaliar a política de mobilidade urbana; prestar os serviços de transporte coletivo urbano, que possuem caráter essencial; e desenvolver institucionalmente os órgãos e entidades responsáveis por sua gestão \cite{lei12587}. Assim, a adesão à política dependerá da iniciativa e concordância de cada prefeitura. O papel do Governo do Estado será o de fomentar essa agenda por meio da construção de parcerias institucionais, do apoio técnico e da coordenação das ações municipais.

Nesse sentido, três secretarias estaduais desempenharão papel estratégico no processo de implementação: a Secretaria de Desenvolvimento Econômico (SDE), responsável por articular a política com os objetivos de crescimento e inclusão produtiva; a Secretaria de Meio Ambiente, Infraestrutura e Logística (SEMIL), que contribuirá com análises e diretrizes para maximizar os impactos ambientais positivos da proposta; e a Secretaria de Transportes Metropolitanos (STM), que atuará como instância técnica de referência, apoiando o desenho da política com base na experiência acumulada na regulação do transporte público na RMSP através da Empresa Metropolitana de Transportes Urbanos (EMTU).
