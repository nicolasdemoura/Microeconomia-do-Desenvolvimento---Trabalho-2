
\section{Desenh
o da Avaliação Experimental de Impacto}
A avaliação de impacto tem como objetivo central verificar, de forma sistemática e rigorosa, se uma determinada política pública gerou efeitos mensuráveis sobre os resultados de interesse, isolando esses efeitos de outras influências externas. No presente caso, busca-se investigar se a implementação da política de Tarifa Zero provoca alterações significativas em três eixos fundamentais: mobilidade urbana, meio ambiente e mercado de trabalho. Especificamente, pretende-se responder às seguintes perguntas:

\begin{itemize} 
    \item A ampliação do subsídio ao transporte coletivo contribui para a redução dos congestionamentos urbanos? 
    \item Há evidências de que o subsídio ao transporte coletivo promove a redução das emissões de gases de efeito estufa? 
    \item A política de Tarifa Zero tem potencial para mitigar fricções no mercado de trabalho, facilitando o acesso ao emprego formal? 
\end{itemize}

\subsection{Dados}

Os impactos da política serão avaliados a partir de variáveis pertencentes a três dimensões analíticas: mobilidade urbana, meio ambiente e mercado de trabalho. Cada uma dessas dimensões será examinada com base em fontes de dados públicas e oficiais, selecionadas por sua regularidade, abrangência e confiabilidade.

\subsubsection{Mobilidade Urbana}
O impacto da política será avaliado com base na taxa de participação dos modais coletivos no total de viagens realizadas no município, assim como no tempo médio de deslocamento considerando todos os tipos de transporte. A proposta é verificar se a iniciativa afeta o uso do transporte público em comparação aos meios individuais, como automóveis e motocicletas.

A literatura aponta que políticas semelhantes costumam estimular o uso do transporte coletivo, o que tende a reduzir o número de veículos nas vias. Com isso, espera-se uma melhor utilização do espaço urbano e uma diminuição nos tempos médios de viagem, à medida que os congestionamentos se tornam menos frequentes.

Para medir esses efeitos, serão utilizados dados da Pesquisa Nacional de Mobilidade Urbana Municipal (Pemob Municipal), realizada anualmente pelo Ministério das Cidades desde 2018. Serão considerados os indicadores 4.7.1 I (“Qual o percentual de viagens realizadas por transporte coletivo?”) e 4.7.1 F (“Considerando todos os modais de transporte, qual o tempo médio das viagens realizadas no município?”). 

Os dados da Pemob Municipal\footnote{Os dados podem ser encontrados em \url{https://www.gov.br/cidades/pt-br/assuntos/mobilidade-urbana/pesquisa-nacional-de-mobilidade-urbana-pemob}.} estão integralmente disponíveis ao público por meio do site do Ministério das Cidades. Embora a pesquisa seja de alta confiabilidade, vemos limitações em sua abrangência, cobrindo apenas 20 dos 630 municípios paulistas elegíveis para o experimento proposto. Para superar essa lacuna, propõe-se a realização de uma pesquisa complementar, replicando a metodologia da Pemob, com o objetivo de coletar dados específicos sobre o tempo médio de deslocamento considerando todos os modais de transporte e a proporção de usuários que optam por transporte público nos demais 610 municípios não contemplados pela pesquisa oficial.  

A coleta será realizada por meio de entrevistas domiciliares em amostras representativas e aleatórias de cada município, utilizando questionários padronizados baseados no modelo "Origem-Destino" da Pemob, com ênfase em informações sobre tempo de viagem, modais utilizados (públicos, privados, bicicletas, caminhada), frequência de deslocamentos e motivações para escolha modal. As informações serão coletadas em três momentos: (i) seis meses antes da implementação da política (linha de base), (ii) seis meses após o início do tratamento (primeira onda de avaliação), e (iii) 12 meses após a implementação (segunda onda), visando capturar efeitos de curto e médio prazo. Adicionalmente, nos municípios de controle, a coleta ocorrerá nos mesmos intervalos, garantindo comparabilidade temporal e minimizando viés sazonal.  

Apesar do rigor metodológico, os dados primários coletados podem apresentar desafios, como viés de autodeclaração (como imprecisões na estimativa de tempo de viagem por parte dos entrevistados ou subnotificação do uso de transporte privado/público, por vieses próprios dos respondentes), além de possíveis distorções por sazonalidade não capturada (variações no tráfego devido a feriados, eventos locais, obras em vias públicas ou acidentes). A dependência de entrevistas domiciliares também introduz riscos de seleção, especialmente em áreas de alta vulnerabilidade, onde o acesso a domicílios pode ser limitado - algo mais comum para regiões mais rurais ou periféricas. Por fim, a replicação da metodologia da Pemob em larga escala pode elevar custos operacionais e demandar treinamento intensivo de entrevistadores para garantir padronização, com risco de inconsistências na aplicação do questionário. 

\subsubsection{Meio-Ambiente}
Outra dimensão relevante da avaliação diz respeito ao impacto ambiental da política, especialmente no que se refere à emissão de gases poluentes. A lógica subjacente é que, à medida que o uso de transportes coletivos aumenta, tende-se a uma redução da emissão de poluentes per capita, mantendo-se os demais fatores constantes. Para isso, serão analisadas as concentrações de dióxido de carbono (CO2), dióxido de nitrogênio (NO2), dióxido de enxofre (SO2) e ozônio (O3).

Aqui existem dois efeitos a serem considerados. O primeiro é um efeito direto, em que o barateamento do transporte público aumenta a quantidade demandada por esse meio, aumentando a quantidade de poluentes emitidos. O segundo é um efeito substituição, em que a redução do uso de automóveis e motocicletas diminui a quantidade de poluentes emitidos. A literatura aponta que o efeito substituição tende a ser maior que o efeito direto, levando a uma redução líquida das emissões \cite{NO_DAM}. 

As informações sobre a qualidade do ar utilizadas na análise são publicadas pela Companhia Ambiental do Estado de São Paulo (CETESB), que mantém um painel de consulta pública com os dados consolidados de suas estações de monitoramento\footnote{Os dados podem ser encontrados em \url{https://cetesb.sp.gov.br/ar/classificacao-de-municipios/}.}. As séries históricas anuais e, em alguns casos, diárias ou horárias de concentração de poluentes estão disponíveis no site da instituição. Por se tratarem de medições ambientais automatizadas ou coletadas por técnicos em pontos fixos de observação, os dados são considerados não identificados, podendo ser utilizados livremente para fins acadêmicos ou de formulação de políticas públicas. 

\subsubsection{Mercado de Trabalho}
A dimensão relacionada ao mercado de trabalho busca investigar possíveis efeitos estruturais da política sobre o acesso ao emprego por parte das camadas mais vulneráveis da população. A literatura sugere que, com a gratuidade no transporte público, indivíduos que dependem exclusivamente desses meios de locomoção passam a ter acesso a oportunidades profissionais anteriormente inviáveis devido aos custos de deslocamento. A implementação da Tarifa Zero, portanto, pode favorecer uma maior integração espacial do mercado de trabalho e contribuir para a redução do desemprego friccional \cite{NO_DAM,BETTER_FIRMS}.

Além das taxas de ocupação e desemprego, a análise também irá considerar a evolução da massa salarial formal como indicador do dinamismo econômico e da ampliação do acesso ao mercado de trabalho. A hipótese é que, ao facilitar a mobilidade, a política possa resultar não apenas em mais vínculos empregatícios, mas também em uma ampliação da folha de pagamento agregada no município, especialmente entre os segmentos de menor renda.

As informações serão obtidas a partir de duas fontes principais: o Cadastro Geral de Empregados e Desempregados (CAGED), com dados mensais sobre admissões, desligamentos e salários de entrada no mercado formal; e a Relação Anual de Informações Sociais (RAIS), que oferece um retrato detalhado da folha de pagamento anual dos estabelecimentos, incluindo remunerações médias por setor, faixa etária e escolaridade. A RAIS será fundamental para avaliar os efeitos da política sobre a estrutura salarial e o nível agregado de renda formal na economia local. 

Tanto o CAGED quanto a RAIS disponibilizam versões públicas de seus microdados, acessíveis por meio dos portais oficiais do Ministério do Trabalho e Emprego e do Ministério da Economia. No caso do CAGED, os microdados mensais apresentam informações detalhadas sobre admissões e desligamentos, sem identificação nominal, mas com variáveis sobre sexo, faixa etária, setor e ocupação. Já a RAIS oferece duas modalidades de acesso: uma base pública não identificada, com variáveis anonimizadas, e uma base identificada, que pode ser acessada mediante solicitação formal, geralmente condicionada à aprovação de um comitê de ética ou à assinatura de termos de confidencialidade. A base não identificada, no entanto, já permite análises detalhadas da evolução da massa salarial, vínculos por setor e distribuição regional dos empregos formais.

\subsection{Desenho Experimental}

A implementação da política será acompanhada de uma estratégia de avaliação de impacto baseada em um experimento de campo. Essa abordagem permite mensurar de forma rigorosa os efeitos da Tarifa Zero sobre indicadores de mobilidade, emprego e meio ambiente. Entretanto, tal desenho levanta considerações éticas relevantes, sobretudo por envolver a concessão da política pública a apenas parte dos municípios em um primeiro momento, enquanto outros permanecem como grupo de comparação.

\subsubsection{Comitê de Ética}

Embora a seleção dos municípios participantes ocorra por meio de randomização — procedimento essencial para garantir a validade estatística da avaliação de impacto —, é necessário atentar para as implicações éticas desse tipo de intervenção. O desenho caracteriza-se como um \textit{field experiment}, em que os agentes afetados, como cidadãos e administrações municipais, serão informados sobre a implementação da política, mas não fornecerão consentimento individual formal. Tal dispensa é compatível com diretrizes consolidadas em ética em pesquisa, desde que se observem critérios internacionalmente reconhecidos: (i) a intervenção não impõe riscos relevantes aos sujeitos; (ii) seus direitos e bem-estar não são comprometidos; e (iii) a exigência de consentimento inviabilizaria a viabilidade prática e escalável da política. No contexto brasileiro, essa lógica é respaldada pela Resolução nº 510, de 7 de abril de 2016, do Conselho Nacional de Saúde, que prevê a possibilidade de dispensa de consentimento livre e esclarecido em pesquisas com dados de interesse público e baixo risco \cite{CNS_2016}.

A proposta será submetida ao Comitê de Ética em Pesquisa competente, em conformidade com o marco regulatório vigente, e sua implementação será guiada pelos princípios da transparência, da responsabilidade institucional e do compromisso com o interesse público. A distribuição aleatória e temporária da política, embora possa gerar percepções de desigualdade \textit{ex post}, representa uma forma equitativa de alocação \textit{ex ante}, assegurando que todos os municípios tenham iguais chances de receber o programa.

Além de orientar a implementação da política, o experimento servirá também como base para a produção de evidências científicas de alta qualidade. Para assegurar a transparência e evitar vieses de publicação, será elaborado um plano de análise prévia (\textit{pre-analysis plan}), que detalhará o desenho experimental, as variáveis de interesse e os métodos estatísticos a serem utilizados. Esse plano será registrado publicamente, antes do início da coleta de dados, em plataformas reconhecidas como o repositório da\textit{ American Economic Association} \cite{AEA_registry}, permitindo o escrutínio público e o compromisso com as estratégias analíticas previamente definidas. Essa prática reforça o rigor metodológico e ético do projeto, promovendo a reprodutibilidade dos resultados e contribuindo para o aprimoramento das políticas públicas baseadas em evidência.

\subsubsection{Randomização}

A política será implementada em 630 municípios do Estado de São Paulo, excluindo-se aqueles que já adotaram políticas similares. Como se trata de uma política pública com custos consideráveis, não é factível, do ponto de vista orçamentário, que o Governo do Estado implemente o programa simultaneamente em todos os municípios. Por esse motivo, a adoção ocorrerá em ondas sucessivas, permitindo a identificação do impacto da política por meio de um desenho experimental com grupos de tratamento e controle.

A divisão em ondas será determinada de forma a maximizar o poder estatístico do experimento, respeitando os limites do orçamento público. A proposta inicial contempla três ondas de implementação, com aproximadamente 210 municípios tratados em cada uma. Essa estratégia de alocação escalonada permite que o grupo de controle sirva de comparação contrafactual para os grupos tratados, o que reforça a validade causal das estimativas, conforme discutido em \citeonline{duflo2008toolkit}.

O poder estatístico do experimento depende da variância do resultado de interesse, do tamanho amostral, e da proporção de unidades tratadas. Em um cenário de aleatorização simples, assumindo homocedasticidade, o \textit{Minimum Detectable Effect} (MDE) pode ser aproximado pela seguinte fórmula:

\begin{equation}
    \text{MDE} = (t_{\kappa} + t_{1 - \alpha/2}) \cdot \sqrt{\frac{\sigma^2}{N}} \cdot \sqrt{\frac{1}{P(1 - P)}},
\end{equation}
em que $t_{\kappa}$ representa o valor crítico do teste para o poder desejado (por exemplo, 0{,}84 para 80\% de poder), $t_{1 - \alpha/2}$ o valor crítico para o nível de significância (por exemplo, 1{,}96 para $\alpha = 0{,}05$), $\sigma^2$ a variância do resultado de interesse, $P$ a proporção de municípios tratados, e $N$ o número total de municípios na amostra. Essa equação orientará a decisão sobre o tamanho ideal dos grupos em cada onda de randomização, equilibrando viabilidade prática e precisão estatística. 

% Por exemplo, \citeonline{duflo2008toolkit} mostram que se considerarmos um cenário que o custo de tratar um município é de $c_t$ e de ter um município como controle é de $c_c$, e um orçamento de $B$ unidades monetárias, a alocação ótima de municípios possui como condição de primeira ordem:
% \begin{equation}
%     \frac{P}{1 - P} = \sqrt{\frac{c_c}{c_t}} 
% \end{equation}
% onde $P$ é a proporção de municípios tratados. Essa condição sugere que, se o custo de tratar um município for maior do que o custo de mantê-lo no controle, a proporção de municípios tratados deve ser menor do que 50\% para minimizar o MDE.

Para aumentar garantir o equilíbrio ex post entre os grupos de tratamento e controle, a randomização será conduzida de forma estratificada, conforme sugerido por \citeonline{duflo2008toolkit}. Os estratos serão definidos com base em três características dos municípios no período pré-intervenção: (i) \textit{renda per capita}; (ii) \textit{população total}; e (iii) \textit{característica urbano-rural}, definida a partir da proporção da população residente em áreas urbanas. A classificação dos municípios nessas dimensões busca capturar heterogeneidade relevante para a implementação e os efeitos da política pública, especialmente no que diz respeito à estrutura produtiva local, à densidade de deslocamentos e ao acesso a serviços públicos.

Serão formados 8 estratos combinando diferentes níveis dessas variáveis, e, dentro de cada estrato, a alocação ao grupo de tratamento será realizada por sorteio aleatório. Esse procedimento não apenas assegura comparabilidade entre os grupos em termos de características socioeconômicas observáveis, como também aumenta o poder estatístico da análise e permite a investigação de efeitos heterogêneos da política ao longo desses eixos \cite{duflo2008toolkit, angrist2009mostly}.

No desenho proposto, antecipamos a ocorrência de conformidade imperfeita unilateral (\textit{one-sided noncompliance}), situação na qual alguns municípios sorteados para o grupo de controle podem, por iniciativa própria, implementar políticas similares à Tarifa Zero durante o período do estudo. Isso é provável através, por exemplo, do efeito John Henry, em que as unidades de controle tentam se igualar às unidades tratadas, buscando evitar a perda de competitividade devido à intervenção. Por outro lado, não esperamos que municípios alocados ao grupo de tratamento deixem de implementar a política, uma vez que sua adesão será viabilizada por meio de apoio técnico e financeiro do Governo do Estado. Esse cenário caracteriza um problema de conformidade unilateral, que pode ser tratado analiticamente recuperando outros parâmetros de interesse como o \textit{Intention-to-Treat} (ITT) e \textit{Local Average Treatment Effect} (LATE), obtido por modelos de variáveis instrumentais, em que a alocação aleatória é utilizada como instrumento para a efetiva adoção da política \cite{duflo2008toolkit,glewwe2022impact,
angrist2009mostly}.

Adicionalmente, não se espera que haja atrito da amostra, uma vez que a unidade de análise é o município e as informações de interesse — como adoção da política, indicadores de mobilidade, gases do efeito estufa e dados socioeconômicos — podem ser obtidas de forma sistemática e contínua a partir de fontes administrativas. 

Por fim, ignoramos a possibilidade de contaminação entre os grupos de tratamento e controle, uma vez que a política será implementada será com relação a transporte intra-municipal, e a gratuidade do transporte público não se estenderá a outros modais, como ônibus intermunicipais ou trens metropolitanos. 

